Os resultados obtidos neste trabalho são iniciais 
do ponto de vista de exploração da 
Lógica Paraconsistente Anotada Evidencial $E\tau$ 
utilizada para o controle dinâmico de sistemas, 
e apresenta-se como promissor o caminho 
associado à técnicas de sistemas adaptativos,
para alteração de parâmetros de controle.

Apesar de não ser o foco do trabalho, 
que ficou restrito a uma implementação inicial 
da LPA$E\tau$ em sistemas de controle,
cabe ainda uma maior investigação do seu uso
juntamente com o controle clássico,
bem como outras tecnicas de controle,
e outros tipos de aplicações.



%Conclui-se que ... 
%como sistema operacional GNU/Linux Debian 8(Jessie), 
%GNOME Shell, 
%Editor de texto e códigos fonte VIM, 
%compilador GCC para ARM (arm-none-eabi-gcc), 
%GNU Make, 
%processador de texto \LaTeX - pdfTEX, 
%pacotes geradores de figuras TikZ, PGF e GNU pic(Groff), 
%gerador de gráficos GNUPlot, 
%teminal de comunicação Minicom e 
%gravador LM4Flash.


