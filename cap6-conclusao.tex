O presente trabalho apresentou uma proposta ousada pela inovação na
utilização da LPA$E\tau$ aplicada em constrole de sistemas,
mesmo que de modo ainda inicial,
procurando conciliar alguns conceitos e contornar outros.
Por se tratar de uma forma até então não explorada,
pode-se perceber algumas possibilidades
tanto em configurações alternativas do controlador como nas
possibilidades de que se apresentam como promissoras.


São as principais contribuições deste trabalho:

\begin{itemize}

\item Apresentação de uma nova proposta para realização do controle dinâmico de sistema utilizando a Lógica Paraconsistente Anotada Evidencial $E\tau$.

\item Aplicação de um método de validação dessa nova proposta baseada em comparação com uma implementação bem estabelecida, aceita e utilizada pelo meio acadêmico e industrial.

\item Aplicação bem sucedida mediante o objetivo e aos requisitos de desempenho do sistema apresentados;

\item Compreensão da LPA$E\tau$ e suas formas de aplicação;

\item Investigação das possibilidades e áreas distintas de aplicação;

\item Aplicação da LPA$E\tau$ em um sistema de controle;

\item Ampliação do conhecimento sobre a LPA$E\tau$ sob uma perspectiva até então não explorada.

\item Possibilitar o início de uma linha de pesquisa tendo como base o estudo da LPA$E\tau$ aplicado ao controle de sistemas;

\item Evidenciar possibilidades de trabalhos futuros.

\end{itemize}


Os resultados obtidos neste trabalho são iniciais do ponto de vista de exploração da Lógica Paraconsistente Anotada Evidencial $E\tau$ utilizada para o controle dinâmico de sistemas, e apresenta-se como promissor o caminho associado à técnicas de sistemas adaptativos, inteligência artificial, para alteração de parâmetros de controle.


\section{Trabalhos futuros}

Como um dos principais resultados do presente trabalho está o apontamento de possíveis caminhos a serem trilhados futuramente, dando prosseguimento à linha de trabalho, ampliando os horizontes, sedimentando os conhecimentos aqui apresentados, corrigindo os posssíveis equívocos e aprofundando conceitos.

Como principais sugestões para trabalhos futuros são citados:

\begin{itemize}

\item Controle de sistemas não lineares: o presente trabalho, por se tratar de uma abordagem inicial, buscou uma aplicação em um sistemas mais simples, para validar os conceitos iniciais, reduzindo as possíveis fontes de complexidade e problemas;

\item Aplicar o controlador LPA$E\tau$ em um sistema de segunda ordem e avaliar as implicações, limitações e potenciais;
  
\item Controle de sistems críticos: aplicar a LPA$E\tau$ em sistemas cuja criticidade é mandatória, exigindo um processamento e tomada de decisão consistente, precisa e de resposta imediata;

\item Utilizar um sistema operacional de tempo real para gerenciar o comportamento do controlador, explorando o viés comportamental da implementação do controlador LPA$E\tau$ em um RTOS tanto \emph{soft} quanto \emph{hard}, com aplicações não críticas e críticas;

\nomenclature{$RTOS$}{$\emph{Real Time Operating Systems}$ - Sistema Operacional de Tempo Real}
  
\item Melhoria da geração do parâmetro $\delta$, utilizando um algorítmo adaptativo, inteligência artificial, ou alguma técnica que permita um melhor ajuste deste valor de correção.
  
\end{itemize}




%O processo de implementação do controlador utilizando a 
%Lógica Paraconsistente Anotada Evidencial $E\tau$
%produziu diversas tentativas, configurações, alterações, 
%sendo que segue aquela que melhor resultado apresentou até
%o devido ponto que se adotou nessa dissertação, 
%não esgotando as formas e tentativas que poderão se seguir 
%no decorrer da pesquisa.

%Apesar de não ser o foco do trabalho, que ficou restrito a uma implementação inicial da LPA$E\tau$ em sistemas de controle, cabe ainda uma maior investigação do seu uso juntamente com o controle clássico,
%bem como outras tecnicas de controle, e outros tipos de aplicações.
