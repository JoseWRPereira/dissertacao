%CONTEXTO: Identifica a grande area de pesquisa e sua importancia;

%LACUNA: (however:contudo, all do:todos fazem) O que precisa ser estudado nesse campo e ainda precisa ser entendido, exclarecido. Local aonde o trabalho está inserido;

%PRÓPOSITO: (this paper describes...)o que foi feito e principal objetivo do artigo;

%METODOLOGIA: Maneira geral falar dos métodos

%RESULTADOS: IMPORTANTÍSSIMO!!! Principal achado, resultado de forma bastante clara;

%CONCLUSOES: Mostrar como que os resutados contribuem para o avanço da grande aréa.

Sistemas de Controle são amplamente utilizados no setor industrial e buscam uma maior eficiência de tempo e energia, mantendo uma qualidade dos processos e do sistema controlado.
% Propósito
O objetivo deste estudo é mostrar uma implementação não convencional de um controle em malha fechada utilizando a Lógica Paraconsistente Anotada Evidencial $E\tau$, de forma a atender requisitos específicos de desempenho em um sistema físico proposto.
% Metodologia
O desenvolvimento de técnicas classificadas como Inteligência Artificial fez surgirem outras opções para o controle de sistemas, contudo ainda há escassez de implementações e testes usando técnicas alternativas.
% Lacuna
% Resultados
%Conclusão		
Os estudos e os resultados iniciais mostram um grande potencial para a implementação e exploração da LPA2v em sistemas de controle, de forma semelhante as técnicas mais difundidas como uma Lógica Fuzzy, Redes Neurais, Controle Adaptativo ou Algorítmo Evolutivo.


