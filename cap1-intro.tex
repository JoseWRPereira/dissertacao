%%%Objetivo ou Finalidade: preocupação em distinguir a característica comum ou as leis gerais que regem determinados eventos.

Tendo em vista que estudos de novas formas de lógicas não clássicas estão em curso, a lógica paraconsistente surge como uma promissora ferramenta para tomada de decisão em diversos campos de aplicação como a robótica, automação industrial, inteligência artificial, logística, controle, entre outras\cite{JoaoInacio}.
 
Segundo o Dr. \citeauthor{DecioKrause}(\citeyear{DecioKrause}), professor e pesquisador do departamento de Filosofia da Universidade Federal de Santa Catarina: "Alguns dos campos mais férteis de aplicação dessas lógicas têm sido a ciência da computação, a engenharia e a medicina." e cita ainda que:
\begin{citacao}{
 "... na inteligência artificial essas lógicas foram usadas a partir da década de 1980 por H. Blair e V. S. Subrahmanian, da Universidade de Siracusa, Estados Unidos, e colaboradores, na elaboração de sistemas para serem utilizados especialmente em medicina." 
}
\end{citacao}

A Lógica Clássica, que utiliza um modelo lógico binário, foi de forma muito natural adaptado ao funcionamento dos transistores utilizados como chave liga/desliga, e este funcionamento embasou toda a tecnologia digital que vemos hoje em dia, baseada em princípios bem definidos e reais. Assim, surge a indagação sobre a utilização de Lógicas Paraconsistentes aplicadas ao mundo real como cita \citeauthor{JISF2011}(\citeyear{JISF2011}):

\begin{citacao}{
Dentro desta percepção, surge a ideia da possibilidade real de um Sistema Lógico Paraconsistente que, assim como na lógica clássica, é um conjunto de axiomas e regras de inferência que objetivam representar formalmente raciocínio válido. Sendo assim, o Sistema Lógico Paraconsistente pode ser representado através de um algoritmo que tem sua utilização como o núcleo de um programa computacional com aplicações diretas em sistemas de Inteligência Artificial.
}
\end{citacao}



Algumas das Lógicas Paraconsistentes ainda não tiveram  uma abordagem prática de sua implementação, ou ainda, tais abordagens são muito escassas, seja com dispositivos simples ou com os mais complexos. 

Visando uma melhor compreensão da Lógica Paraconsistente, e vislumbrando sua utilização em controle de sistemas dinâmicos utilizando um ramo denominado Lógica Paraconsistente Anotada de Anotação com dois valores (LPA2v), pressupõe-se um estudo de uma aplicação inicial como forma de desbravar uma nova possibilidade da utilização de um algoritmo que vem sendo aplicado com sucesso em Inteligência Artificial no segmento de Controle.






%%%%%%%%%%%%%%%%%%%%%%%%%%%%%%%%%%%%%%%%%%%%%%%%%%%%%%%%%%%%
\section{Hipótese e Relevância do Trabalho}
%%%%%%%%%%%%%%%%%%%%%%%%%%%%%%%%%%%%%%%%%%%%%%%%%%%%%%%%%%%%
A Lógica Paraconsistente Anotada de anotação com dois valores (LPA2v) pode ser utilizada para o controle de sistemas dinâmicos, hipótese esta que confirmada pode elevar ainda mais a sua relevância e elencar mais uma alternativa para aplicações técnicas e científicas. 

A lógica paraconsistente vem ganhando relevância e adeptos principalmente a partir do final da década de 90 do século XX, quando houve o Primeiro Congresso Mundial sobre Paraconsistência em Gent na Bélgica em 1997, no ano 2000 o segundo congresso realizado em São Sebastião, São Paulo e o terceiro em Toulouse, França em julho de 2003, atraindo cada vez mais pesquisadores interessados de diversos centros de pesquisa do mundo \cite{DecioKrause}. 

Em meados de setembro de 2016, aconteceu o pela primeira vez no Brasil a XVI Conferência Internacional de Lógica: \texttt{Tendências da Lógica} (\emph{Trends In Logic XVI - Studia Logica International Conference}) \cite{trendsinlogic}, realizada pelo Centro de Lógica, Epistemologia e História da Ciência (CLE) da Universidade Estadual de Campinas, que reuniu estudiosos brasileiros e de diversos países com trabalhos e apresentações sob o tema: Consistência, Contradição, Paraconsistência e Racioncínio (\emph{Consistency, Contradiction, Paraconsistency, and Reasoning}).

Atualmente as pesquisas estão focadas no estudo da aplicação da Lógica Paraconsistente, e ganhar espaço no universo técnico e científico, contribuindo com uma nova e eficiente forma de trabalho.





%%%%%%%%%%%%%%%%%%%%%%%%%%%%%%%%%%%%%%%%%%%%%%%%%%%%%%%%%%%%
\section{Objetivo Geral}
%%%%%%%%%%%%%%%%%%%%%%%%%%%%%%%%%%%%%%%%%%%%%%%%%%%%%%%%%%%%
%Objeto, subdividido em:
%Material: aquilo que pretende estudar, analisar, interpretar ou verificar de modo geral.
Estudar a LPA2v e desenvolver um algoritmo que possa ser embarcado para atuar no controle dinâmico de um sistema físico.




%%%%%%%%%%%%%%%%%%%%%%%%%%%%%%%%%%%%%%%%%%%%%%%%%%%%%%%%%%%%
\section{Objetivos Específicos}
%%%%%%%%%%%%%%%%%%%%%%%%%%%%%%%%%%%%%%%%%%%%%%%%%%%%%%%%%%%%
%Formal:enfoque especial, 
%	em face de diversas ciências que possuem o mesmo objetivo material.


Realizar a construção de um sistema físico para o controle de velocidade em um motor DC, de modo a utilizá-lo para a realização dos ensaios.

Implementar o algorítmo da LPA2v proposto na literatura de referência, codificando-o e criando uma biblioteca para sua utilização junto ao sistema físico.

Desenvolver a malha de controle do sistema físico proposto utilizando o algorítmo da LPA2v.




%%%%%%%%%%%%%%%%%%%%%%%%%%%%%%%%%%%%%%%%%%%%%%%%%%%%%%%%%%%%
\section{Justificativa}
%%%%%%%%%%%%%%%%%%%%%%%%%%%%%%%%%%%%%%%%%%%%%%%%%%%%%%%%%%%%

%Função:aperfeiçoamento, 
%	através de crescente acervo de conhecimento, 
%	da relação do homem com o seu mundo.


A função primordial do presente trabalho é realizar uma análise da implementação de uma lógica não-convencional, 
contribuindo para a ampliação do conhecimento em uma nova forma de lidar com o mundo e
gerar aplicações em uma área ainda pouco explorada pela LPA2v, o controle de sistemas. 


A visão aristotélica cunhou a forma lógica de lidar com o mundo, 
estabelecendo regras que permearam a história até o presente momento,
mas que, no século XX foram questionadas, procurando-se novas formas e ferramentas para tratar de questões que fogem das regras vigentes. 
A Lógica Paraconsistente é uma das ferramentas que permite o tratamento de contradições e incertezas,
que estão além dos limites da lógica clássica.

A LPA2v é a uma vertente da Lógica Paraconsistente que vem sendo explorada para finalidades práticas, 
tais como o reconhecimento de padrões em banco de dados, tomada de decisão e tratamento de incertezas em sistemas robóticos e logísticos, 
mas todas as áreas com uma abordagem ligada à Inteligência Artificial ou ao controle discreto do processo, 
ainda há escassez de trabalhos no controle contínuo de sistemas dinâmicos. 


A análise da implementação da LPA2v no universo das lógicas não-convencionais implica em possibilitar uma nova forma de controle de sistemas, 
sua definição permite um embasamento para criar novas possibilidades de seu uso,
e ajudar a sedimentar a nova ferramenta no meio acadêmico. 



%%%%%%%%%%%%%%%%%%%%%%%%%%%%%%%%%%%%%%%%%%%%%%%%%%%%%%%%%%%%
%\section{Limitações da pesquisa}
%%%%%%%%%%%%%%%%%%%%%%%%%%%%%%%%%%%%%%%%%%%%%%%%%%%%%%%%%%%%

%%% Não colocar o elemento TEMPO! 
%%% Colocar até onde vai a pesquisa e o que ela não abordará.

%Este estudo limitar-se-á ao estudo da LPA2v, e possivelmente uma pequena Rede de Análise Paraconsistente, voltado à Inteligência Artificial, mas sempre com o objetivo de realizar o controle dinâmico do sistema proposto.

%Não faz parte a abordagem do modo clássico de controle PID, Avanço de Fase, ou mesmo Controle Moderno, assim como a abordagem e especificações do controlador ou mesmo da linguagem utilizada para a montagem do protótipo.


