
Tendo em vista que estudos de novas formas de lógicas não clássicas estão em curso, a lógica paraconsistente surge como uma promissora ferramenta para tomada de decisão em diversos campos de aplicação como a robótica, automação industrial, inteligência artificial, logística, controle, entre outras\cite{JoaoInacio}.
 
Segundo o Dr. \citeauthor{DecioKrause}(\citeyear{DecioKrause}), professor e pesquisador do departamento de Filosofia da Universidade Federal de Santa Catarina: "Alguns dos campos mais férteis de aplicação dessas lógicas têm sido a ciência da computação, a engenharia e a medicina." e cita ainda que:
\begin{citacao}{
 "... na inteligência artificial essas lógicas foram usadas a partir da década de 1980 por H. Blair e V. S. Subrahmanian, da Universidade de Siracusa, Estados Unidos, e colaboradores, na elaboração de sistemas para serem utilizados especialmente em medicina." 
}
\end{citacao}

A Lógica Clássica, que utiliza um modelo lógico binário, foi de forma muito natural adaptado ao funcionamento dos transistores utilizados como chave liga/desliga, e este funcionamento embasou toda a tecnologia digital que vemos hoje em dia, baseada em princípios bem definidos e reais. Assim, surge a indagação sobre a utilização de Lógicas Paraconsistentes aplicadas ao mundo real como cita \citeauthor{JISF2011}(\citeyear{JISF2011}):

\begin{citacao}{
Dentro desta percepção, surge a ideia da possibilidade real de um Sistema Lógico Paraconsistente que, assim como na lógica clássica, é um conjunto de axiomas e regras de inferência que objetivam representar formalmente raciocínio válido. Sendo assim, o Sistema Lógico Paraconsistente pode ser representado através de um algoritmo que tem sua utilização como o núcleo de um programa computacional com aplicações diretas em sistemas de Inteligência Artificial.
}
\end{citacao}



Algumas das Lógicas Paraconsistentes ainda não tiveram  uma abordagem prática de sua implementação, ou ainda, tais abordagens são muito escassas, seja com dispositivos simples ou com os mais complexos. 

Visando uma melhor compreensão da Lógica Paraconsistente, 
e vislumbrando sua utilização em controle de sistemas dinâmicos 
utilizando um ramo denominado 
Lógica Paraconsistente Anotada Evidencial $E\tau$, 
pressupõe-se um estudo de uma aplicação inicial 
como forma de desbravar uma nova possibilidade da utilização 
de uma lógica que vem sendo aplicado com sucesso em 
Inteligência Artificial no segmento de Controle.







%%%%%%%%%%%%%%%%%%%%%%%%%%%%%%%%%%%%%%%%%%%%%%%%%%%%%%%%%%%%
%\section{Justificativa}
%%%%%%%%%%%%%%%%%%%%%%%%%%%%%%%%%%%%%%%%%%%%%%%%%%%%%%%%%%%%

%Função:aperfeiçoamento, 
%	através de crescente acervo de conhecimento, 
%	da relação do homem com o seu mundo.



%%%%%%%%%%%%%%%%%%%%%%%%%%%%%%%%%%%%%%%%%%%%%%%%%%%%%%%%%%%%
\section{Objetivo Geral}
%%%%%%%%%%%%%%%%%%%%%%%%%%%%%%%%%%%%%%%%%%%%%%%%%%%%%%%%%%%%
%Objeto, subdividido em:
%Material: aquilo que pretende estudar, analisar, interpretar ou verificar de modo geral.
O objetivo geral do presente trabalho é 
realizar uma análise e implementação da LPA$E\tau$
como uma lógica não-convencional 
em um sistema embarcado para controle dinâmico de
um motor de corrente contínua,
contribuindo para a ampliação do conhecimento em 
uma nova forma de lidar com o mundo e
gerar aplicações nessa área ainda pouco explorada. 




%%%%%%%%%%%%%%%%%%%%%%%%%%%%%%%%%%%%%%%%%%%%%%%%%%%%%%%%%%%%
\subsection{Objetivos Específicos}
%%%%%%%%%%%%%%%%%%%%%%%%%%%%%%%%%%%%%%%%%%%%%%%%%%%%%%%%%%%%

Estudar a LPA$E\tau$ e desenvolver um algoritmo 
que possa ser embarcado para 
atuar no controle dinâmico de um sistema físico.


Realizar a construção de um sistema físico para 
o controle de velocidade em um motor de corrente contínua, 
de modo a utilizá-lo para a realização dos ensaios.


Desenvolver a malha de controle do sistema físico proposto utilizando o algorítmo da $E\tau$.



%%%%%%%%%%%%%%%%%%%%%%%%%%%%%%%%%%%%%%%%%%%%%%%%%%%%%%%%%%%%
%\section{Limitações da pesquisa}
%%%%%%%%%%%%%%%%%%%%%%%%%%%%%%%%%%%%%%%%%%%%%%%%%%%%%%%%%%%%

%%% Não colocar o elemento TEMPO! 
%%% Colocar até onde vai a pesquisa e o que ela não abordará.

%Este estudo limitar-se-á ao estudo da LPA2v, e possivelmente uma pequena Rede de Análise Paraconsistente, voltado à Inteligência Artificial, mas sempre com o objetivo de realizar o controle dinâmico do sistema proposto.

%Não faz parte a abordagem do modo clássico de controle PID, Avanço de Fase, ou mesmo Controle Moderno, assim como a abordagem e especificações do controlador ou mesmo da linguagem utilizada para a montagem do protótipo.



%%%%%%%%%%%%%%%%%%%%%%%%%%%%%%%%%%%%%%%%%%%%%%%%%%%%%%%%%%%%
%\section{Organização da Dissertação}
%%%%%%%%%%%%%%%%%%%%%%%%%%%%%%%%%%%%%%%%%%%%%%%%%%%%%%%%%%%%





