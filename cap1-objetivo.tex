
Tendo em vista que estudos de novas formas de lógicas não clássicas estão em curso, a lógica paraconsistente surge como uma promissora ferramenta para tomada de decisão em diversos campos de aplicação como a robótica, automação industrial, inteligência artificial, logística, controle, entre outras\cite{JoaoInacio}.
 
Segundo o Dr. \citeauthor{DecioKrause}(\citeyear{DecioKrause}), professor e pesquisador do departamento de Filosofia da Universidade Federal de Santa Catarina: "Alguns dos campos mais férteis de aplicação dessas lógicas têm sido a ciência da computação, a engenharia e a medicina." e cita ainda que:
\begin{citacao}{
 "... na inteligência artificial essas lógicas foram usadas a partir da década de 1980 por H. Blair e V. S. Subrahmanian, da Universidade de Siracusa, Estados Unidos, e colaboradores, na elaboração de sistemas para serem utilizados especialmente em medicina." 
}
\end{citacao}

A Lógica Clássica, que utiliza um modelo lógico binário, foi de forma muito natural adaptado ao funcionamento dos transistores utilizados como chave liga/desliga, e este funcionamento embasou toda a tecnologia digital que vemos hoje em dia, baseada em princípios bem definidos e reais. Assim, surge a indagação sobre a utilização de Lógicas Paraconsistentes aplicadas ao mundo real como cita \citeauthor{JISF2011}(\citeyear{JISF2011}):

\begin{citacao}{
Dentro desta percepção, surge a ideia da possibilidade real de um Sistema Lógico Paraconsistente que, assim como na lógica clássica, é um conjunto de axiomas e regras de inferência que objetivam representar formalmente raciocínio válido. Sendo assim, o Sistema Lógico Paraconsistente pode ser representado através de um algoritmo que tem sua utilização como o núcleo de um programa computacional com aplicações diretas em sistemas de Inteligência Artificial.
}
\end{citacao}



Algumas das Lógicas Paraconsistentes ainda não tiveram  uma abordagem prática de sua implementação, ou ainda, tais abordagens são muito escassas, seja com dispositivos simples ou com os mais complexos. 

Visando uma melhor compreensão da Lógica Paraconsistente, 
e vislumbrando sua utilização em controle de sistemas dinâmicos 
utilizando um ramo denominado 
Lógica Paraconsistente Anotada Evidencial $E\tau$ (LPA$E\tau$), 
pressupõe-se um estudo de uma aplicação inicial 
como forma de desbravar uma nova possibilidade da utilização 
de uma lógica que vem sendo aplicado com sucesso em 
Inteligência Artificial no segmento de Controle.




%%%%%%%%%%%%%%%%%%%%%%%%%%%%%%%%%%%%%%%%%%%%%%%%%%%%%%%%%%%%
\section{A LPA$E\tau$ e suas aplicações}
%%%%%%%%%%%%%%%%%%%%%%%%%%%%%%%%%%%%%%%%%%%%%%%%%%%%%%%%%%%%

Os Sistemas Inteligentes estão cada vez mais presentes em diversas aplicações modernas, e segundo \citeauthor{JISF2011}(\citeyear{JISF2011}), há um predomínio de Lógicas Não-Clássicas no suporte à tomada de decisão desses sistemas. O sucesso na aplicação de técnicas como a Lógica $E\tau$, que é uma extensão da Lógica Paraconsistente Anotada, se dá em grande medida pelo uso de algorítmos baseados em estudos dos reticulados representativos e efetiva tradução matemática gerando um modelo eficiente aplicado em situações reais.

Assumindo que a lógica filosófica trata da descrição formal da linguagem natural e define a sua estrutura de declaração, então, sendo encontrada a linguagem adequada é possível traduzir o raciocínio formal em LPA, modelando raciocínios com a possibilidade de tratar contradições ou incoerências, e trabalhar com situações reais, da mesma forma que o Modelo Clássico, que aplica regras computacionais, a Lógica $E\tau$ possui um conjunto de axiomas e regras de inferência que possibilitam um raciocínio válido em situações reais.


\subsubsection{Robótica}

Os algorítmos baseados no estudo do Reticulado Representativo gera Estados Lógicos Paraconsistentes através da descrição do algorítmo Para-Analisador da LPA$E\tau$, possibilitando que o sistema receba informação através dos graus de evidência ($\mu, \lambda$), processe os graus de certeza e contradição ($G_c$ e $G_{ct}$) e chegue a uma conclusão, de alta contradição e busque por mais dados ou um alto grau de certeza, que de um modo geral, implica em tomar uma ação. 

Os Graus de Certeza e Contradição podem gerar um Grau de Certeza Real, que pode servir de entrada para outra célula ou Nó de Análise Paraconsistente (NAP), possibilitando uma rede de análises para a tomada de decisão, como apresentado na construção e aperfeiçoamentos realizados no Robô Emmy, \cite{JoaoInacio}\cite{ClaudioRodrigoTorres} desenvolvendo e aplicando tais técnicas ao sistema de movimentação.


\subsubsection{Engenharia de Produção}

A LPA$E\tau$ pode ser aplicada em diversas áreas sendo um outro exemplo a sua aplicação na área de engenharia de produção, como é mostrado no artigo de \citeauthor{FabioIsraelJair}(\citeyear{FabioIsraelJair}), que mostram um estudo para avaliação do projeto de uma fábrica, como são selecionadas as variáveis relevantes, ou fatores, como são chamados, níveis de exigência para tomada de decisão, 
atribuição de pesos aos fatores de decisão, para obtenção dos graus de crença e descrença. 
Construção de uma base de dados, sua pesquisa e obtenção dos resultados, análise e fidedignidade utilizando um Método de Análise pelo Baricentro.


\subsubsection{Logística}

No segmento logístico pode-se citar a dissertação do Profº Me. Vander Célio Nunes (\cite{Vander}), que aplicou a Lógica $E\tau$ ao processo de paletização através da medição de peças e do tratamento de incertezas relacionadas a possibilidade de seu depósito ou encaixe na pilha de palets, levando à otimização de cargas armazenadas em um determinado espaço. 
O seu trabalho, utilizando uma célula de manufatura com um braço robótico industrial, permite a extrapolação da sua aplicação para portos e armazens de containers.



\subsubsection{Medicina}

Em aplicações de apoio à medicina através de algoritmos para auxílio de diagnóstico de patologias como em \citeauthor{MauricioCM}(\citeyear{MauricioCM}), onde a Lógica Paraconsistente é aplicada na análise de mamografias.



\subsubsection{Sistema de Controle Híbrido}

No segmento de controle, a Lógica $E\tau$ é utilizada em conjunto com um sistema Proporcional-Integral - PI de modo que as ações convencionais são executadas pelo bloco PI, mas são estruturadas utilizando a Lógica $E\tau$ no tratamento dos sinais externos. 
A implementação é feita por \citeauthor{Marcelo}(\citeyear{Marcelo}) em uma planta de controle de nível e um controlador lógico programável. 
O sistema Híbrido é posto em operação e comparado com técnicas consagradas como o controle puramente PI, ajustado com o método de Ziegler-Nichols e com o método interno do controlador. 




%%%%%%%%%%%%%%%%%%%%%%%%%%%%%%%%%%%%%%%%%%%%%%%%%%%%%%%%%%%%
%\section{Justificativa}
%%%%%%%%%%%%%%%%%%%%%%%%%%%%%%%%%%%%%%%%%%%%%%%%%%%%%%%%%%%%

%Função:aperfeiçoamento, 
%	através de crescente acervo de conhecimento, 
%	da relação do homem com o seu mundo.



%%%%%%%%%%%%%%%%%%%%%%%%%%%%%%%%%%%%%%%%%%%%%%%%%%%%%%%%%%%%
\section{Objetivo Geral}
%%%%%%%%%%%%%%%%%%%%%%%%%%%%%%%%%%%%%%%%%%%%%%%%%%%%%%%%%%%%

O objetivo geral do presente trabalho é realizar uma análise e implementação da LPA$E\tau$, como uma lógica não-convencional, em um sistema embarcado para controle dinâmico de um motor de corrente contínua.


Contribuir para a ampliação do conhecimento em uma nova forma de lidar com o mundo, afim de possibilitar a geração de novas aplicações nessa área ainda pouco explorada. 




%%%%%%%%%%%%%%%%%%%%%%%%%%%%%%%%%%%%%%%%%%%%%%%%%%%%%%%%%%%%
\subsection{Objetivos Específicos}
%%%%%%%%%%%%%%%%%%%%%%%%%%%%%%%%%%%%%%%%%%%%%%%%%%%%%%%%%%%%

Estudar a LPA$E\tau$ e desenvolver um algoritmo 
que possa ser embarcado para 
atuar no controle dinâmico de um sistema físico.


Realizar a construção de um sistema físico bem como a malha para 
o controle de velocidade em um motor de corrente contínua, 
de modo a utilizá-lo para a realização dos ensaios utilizando um
algorítmo da LPA$E\tau$.




\section{Relevância do trabalho}

A aplicação da LPA$E\tau$ é ampla e possui abordagens bem sucedidas em
diversas áreas do conhecimento, assim o presente trabalho vem com a
proposta de dar início à pesquisa de sua aplicação em
sistemas de controle, desbravando um caminho ainda não explorado,
mas com a tranquilidade de que são os primeiros passos na união dessas
áreas.

A maior relevância do trabalho está no fato de poder mostrar e balisar
um novo caminho para futuros trabalhos, expondo pontos positivos, dificuldades
iniciais e possibilidades para se trabalhar com a LPA$E\tau$ em
controle de sistemas.




