%CONTEXT: Identifies the large area of ​​research and its importance;

% LACUNA: (however: however, all do: all do)
%What needs to be studied in this field and still needs to be
%understood, clarified.
%Place where the work is inserted;

% PRÓPOSITO: (this paper describes ...) what was done and main objective of the article;

% METHODOLOGY: General way of speaking of methods

% RESULTS: IMPORTANT! Main finding, result quite clear;

% CONCLUSIONS: Show how the results contribute to the advancement of the great area.

% Purpose
The objective of this study is to show an unconventional implementation
of a closed-loop control using the Paraconsistent 
Annotated Logic Evidential $E\tau$ (PAL $E\tau$)
in order to meet specific performance requirements in a proposed physical system.
Control systems are widely used in the industrial sector and
seek efficiency of time and energy,
while maintaining a quality of processes and the controlled system.
% Gap
The development of techniques classified as Intelligence
Artificial has given rise to other options for the control of systems,
however there is still a shortage of implementations and tests using alternative techniques.
% Methodology
For this purpose a survey of the mathematical model of this
physical system, which is used as a reference for the implementation of a
with a PI controller,
as well as system performance requirements,
which guide the implementation of roposing to use PAL $E\tau$,
thus enabling its validation by comparison between the results of the controllers.
% Results
For the system tested, the results between PI controllers and
PAL$E\tau$ were equivalent, with the rise time as
highlight, because for the tolerance window of $5\%$,
value was reached at $ t = 2.5s $ with the PI controller and 
$ t < 1.25s$ with the PAL$E\tau$ driver. 
The results still showed some
possibilities for future improvements, which
more complex systems, such as second-order systems, and
systems without the need to know
control it, adapting its parameters to the occasion after some
learning rounds.
%Conclusion
Initial studies and results show great potential for
implementation and exploitation of PAL$E\tau$ in control systems, in a
similar to the most widespread techniques such as a Fuzzy Logic, Networks
Neural Networks, Adaptive Control, Evolutionary Algorithm, Intelligence
Artificial and Machine Learning, including
using them as support for generation of control parameters.
